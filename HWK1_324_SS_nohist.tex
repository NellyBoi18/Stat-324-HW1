% Options for packages loaded elsewhere
\PassOptionsToPackage{unicode}{hyperref}
\PassOptionsToPackage{hyphens}{url}
%
\documentclass[
]{article}
\usepackage{amsmath,amssymb}
\usepackage{lmodern}
\usepackage{iftex}
\ifPDFTeX
  \usepackage[T1]{fontenc}
  \usepackage[utf8]{inputenc}
  \usepackage{textcomp} % provide euro and other symbols
\else % if luatex or xetex
  \usepackage{unicode-math}
  \defaultfontfeatures{Scale=MatchLowercase}
  \defaultfontfeatures[\rmfamily]{Ligatures=TeX,Scale=1}
\fi
% Use upquote if available, for straight quotes in verbatim environments
\IfFileExists{upquote.sty}{\usepackage{upquote}}{}
\IfFileExists{microtype.sty}{% use microtype if available
  \usepackage[]{microtype}
  \UseMicrotypeSet[protrusion]{basicmath} % disable protrusion for tt fonts
}{}
\makeatletter
\@ifundefined{KOMAClassName}{% if non-KOMA class
  \IfFileExists{parskip.sty}{%
    \usepackage{parskip}
  }{% else
    \setlength{\parindent}{0pt}
    \setlength{\parskip}{6pt plus 2pt minus 1pt}}
}{% if KOMA class
  \KOMAoptions{parskip=half}}
\makeatother
\usepackage{xcolor}
\usepackage[margin=1in]{geometry}
\usepackage{longtable,booktabs,array}
\usepackage{calc} % for calculating minipage widths
% Correct order of tables after \paragraph or \subparagraph
\usepackage{etoolbox}
\makeatletter
\patchcmd\longtable{\par}{\if@noskipsec\mbox{}\fi\par}{}{}
\makeatother
% Allow footnotes in longtable head/foot
\IfFileExists{footnotehyper.sty}{\usepackage{footnotehyper}}{\usepackage{footnote}}
\makesavenoteenv{longtable}
\usepackage{graphicx}
\makeatletter
\def\maxwidth{\ifdim\Gin@nat@width>\linewidth\linewidth\else\Gin@nat@width\fi}
\def\maxheight{\ifdim\Gin@nat@height>\textheight\textheight\else\Gin@nat@height\fi}
\makeatother
% Scale images if necessary, so that they will not overflow the page
% margins by default, and it is still possible to overwrite the defaults
% using explicit options in \includegraphics[width, height, ...]{}
\setkeys{Gin}{width=\maxwidth,height=\maxheight,keepaspectratio}
% Set default figure placement to htbp
\makeatletter
\def\fps@figure{htbp}
\makeatother
\setlength{\emergencystretch}{3em} % prevent overfull lines
\providecommand{\tightlist}{%
  \setlength{\itemsep}{0pt}\setlength{\parskip}{0pt}}
\setcounter{secnumdepth}{-\maxdimen} % remove section numbering
\ifLuaTeX
  \usepackage{selnolig}  % disable illegal ligatures
\fi
\IfFileExists{bookmark.sty}{\usepackage{bookmark}}{\usepackage{hyperref}}
\IfFileExists{xurl.sty}{\usepackage{xurl}}{} % add URL line breaks if available
\urlstyle{same} % disable monospaced font for URLs
\hypersetup{
  pdftitle={Stats 324 Homework 1 Due Wednesday Feb 1st 9 am},
  pdfauthor={Nelson Lu},
  hidelinks,
  pdfcreator={LaTeX via pandoc}}

\title{Stats 324 Homework 1 Due Wednesday Feb 1st 9 am}
\author{Nelson Lu}
\date{}

\begin{document}
\maketitle

*Submit your homework to Canvas by the due date and time. Email your
lecturer if you have extenuating circumstances and need to request an
extension.

*If an exercise asks you to use R, include a copy of the code and
output. Please edit your code and output to be only the relevant
portions.

*If a problem does not specify how to compute the answer, you many use
any appropriate method. I may ask you to use R or use manually
calculations on your exams, so practice accordingly.

*You must include an explanation and/or intermediate calculations for an
exercise to be complete.

*Be sure to submit the HWK1 Autograde Quiz which will give you
\textasciitilde20 of your 40 accuracy points.

*50 points total: 40 points accuracy, and 10 points completion

\hypertarget{basics-of-statistics-and-summarizing-data-numerically-and-graphically-i}{%
\subsection{Basics of Statistics and Summarizing Data Numerically and
Graphically
(I)}\label{basics-of-statistics-and-summarizing-data-numerically-and-graphically-i}}

\textbf{Exercise 1}. A number of individuals are interested in the
proportion of citizens within a county who will vote to use tax money to
upgrade a professional baseball stadium in the upcoming vote. Consider
the following methods:

The \textbf{Baseball Team Owner} surveyed 8,000 people attending one of
the baseball games held in the stadium. Seventy eight percent (78\%) of
respondents said they supported the use of tax money to upgrade the
stadium.

The \textbf{Pollster} generated 1,000 random numbers between 1-52,661
(number of county voters in last election) and surveyed the 1,000
citizens who corresponded to those numbers on the voting roll. Forty
three percent (43\%) of respondents said they supported the use of tax
money to upgrade the stadium.

\begin{quote}
\begin{enumerate}
\def\labelenumi{\alph{enumi}.}
\tightlist
\item
  What is the population of interest? What is the parameter of interest?
  Will this parameter ever be calculated?
\end{enumerate}
\end{quote}

\begin{quote}
\begin{enumerate}
\def\labelenumi{\alph{enumi}.}
\setcounter{enumi}{1}
\tightlist
\item
  What were the sample sizes used and statistics calculated from those
  samples? Are these simple random samples from the population of
  interest?
\end{enumerate}
\end{quote}

\begin{quote}
\begin{enumerate}
\def\labelenumi{\alph{enumi}.}
\setcounter{enumi}{2}
\tightlist
\item
  The baseball team owner claims that the survey done at the baseball
  stadium will better predict the voting outcome because the sample size
  was much larger. What is your response?
\end{enumerate}
\end{quote}

\vspace{1cm}

\textbf{Exercise 2}. There are 12 numbers in a sample, and the mean is
\(\bar{x}=24\). The minimum of the sample is accidentally changed from
11.9 to 1.19.

\begin{quote}
\begin{enumerate}
\def\labelenumi{\alph{enumi}.}
\tightlist
\item
  Is it possible to determine the direction in which (increase/decrease)
  the mean (\(\bar{x}\))changes? Or how much the mean changes? If so, by
  how much does it change? If not, why not?
\end{enumerate}
\end{quote}

\begin{quote}
\begin{enumerate}
\def\labelenumi{\alph{enumi}.}
\setcounter{enumi}{1}
\tightlist
\item
  Is it possible to determine the direction in which the median changes?
  Or how much the median changes? If so, by how much does it change? If
  not, why not?
\end{enumerate}
\end{quote}

\begin{quote}
\begin{enumerate}
\def\labelenumi{\alph{enumi}.}
\setcounter{enumi}{2}
\tightlist
\item
  Is it possible to predict the direction in which the standard
  deviation changes? If so, does it get larger or smaller? If not, why
  not? Describe why it is difficult to predict by how much the standard
  deviation will change in this case.
\end{enumerate}
\end{quote}

\vspace{1cm}

\textbf{Exercise 3:} After manufacture, computer disks are tested for
errors. The table below tabulates the number of errors detected on each
of the 100 disks produced in a day.

\begin{longtable}[]{@{}ll@{}}
\toprule()
Number of Defects & Number of Disks \\
\midrule()
\endhead
0 & 41 \\
1 & 31 \\
2 & 15 \\
3 & 8 \\
4 & 5 \\
\bottomrule()
\end{longtable}

\begin{quote}
\begin{enumerate}
\def\labelenumi{\alph{enumi}.}
\tightlist
\item
  Describe the type of data that is being recorded about the sample of
  100 disks, being as specific as possible.
\end{enumerate}
\end{quote}

\begin{quote}
\begin{enumerate}
\def\labelenumi{\alph{enumi}.}
\setcounter{enumi}{1}
\tightlist
\item
  A frequency histogram showing the number of errors on the 100 disks is
  given below. Write the R code to produce this frequency histogram. Be
  sure to create useful labels. Hints: use the rep() function to define
  your defect data. Also use ylim and breaks to format your graph.
\end{enumerate}
\end{quote}

\begin{quote}
\begin{enumerate}
\def\labelenumi{\alph{enumi}.}
\setcounter{enumi}{2}
\tightlist
\item
  What is the shape of the histogram for the number of defects observed
  in this sample? Why does that make sense in the context of the
  question?
\end{enumerate}
\end{quote}

\begin{quote}
\begin{enumerate}
\def\labelenumi{\alph{enumi}.}
\setcounter{enumi}{3}
\tightlist
\item
  Calculate the mean and median number of errors detected on the 100
  disks by hand and with R. How do the mean and median values compare
  and is that consistent with what we would guess based on the shape?
\end{enumerate}
\end{quote}

\begin{quote}
\begin{enumerate}
\def\labelenumi{\alph{enumi}.}
\setcounter{enumi}{4}
\tightlist
\item
  Calculate the sample standard deviation ``by hand'' and using R. Are
  the values consistent between the two methods? How would our
  calculation differ if instead we know that these 100 values were the
  whole population?
\end{enumerate}
\end{quote}

\begin{quote}
\begin{enumerate}
\def\labelenumi{\alph{enumi}.}
\setcounter{enumi}{5}
\tightlist
\item
  Construct a boxplot for the number of errors data using R with helpful
  labels. Explain how the shape of the data (identified in (c) can be
  seen from the boxplot.
\end{enumerate}
\end{quote}

\begin{quote}
\begin{enumerate}
\def\labelenumi{\alph{enumi}.}
\setcounter{enumi}{6}
\tightlist
\item
  Explain why the histogram is better able to show the discrete nature
  of the data than a boxplot.
\end{enumerate}
\end{quote}

\end{document}
